\subsection{Assumptions and Constraints}

\paragraph{Assumptions}
\newcommand{\DI}{\textbf{D1}: Position of the operator at next time instant is known}
\newcommand{\DII}{\textbf{D2}: There is no enforced time limit on the time that operator and robot stay still at their workstations}
\newcommand{\DIII}{\textbf{D3}: There is no enforced time limit on the time that operator and robot take to move between their workstations}
\newcommand{\DIV}{\textbf{D4}: }
\newcommand{\DV}{\textbf{D5}: }
\newcommand{\DVI}{\textbf{D6}: }
\newcommand{\DVII}{\textbf{D7}: }
\newcommand{\DVIII}{\textbf{D8}: }
\newcommand{\DIX}{\textbf{D9}: }
\newcommand{\DX}{\textbf{D10}: }
\newcommand{\DXI}{\textbf{D11}: }

The following assumptions have been taken into account while designing the model:
\newline\begin{itemize}
	\item  \DI
	\item  \DII
	\item  \DIII
%	\item  \DIV
%	\item  \DV
%	\item  \DVI
\end{itemize}

The domain assumption D1 means that the sensors present on the grid are not only able to detect if the operator is moving, but also \textit{where} it is going. This enables the robot controller to move quite efficiently in cells that are around the operator. This assumption could be removed but, in order to keep the safety property true, every instance in the controller of \textit{next(operatorIn(cell))}) would have to be replaced with a quantification over all the cells that the operator could potentially reach at the next time instant.
\newpage

